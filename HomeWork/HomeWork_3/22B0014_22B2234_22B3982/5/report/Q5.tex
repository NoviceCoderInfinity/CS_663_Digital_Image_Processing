\documentclass{article}
\usepackage{helvet}


\usepackage{cite}
\usepackage{amsmath,amssymb,amsfonts}
\usepackage{algorithmic}
\usepackage{graphicx}
\usepackage{textcomp}
\usepackage{xcolor}
\usepackage{hyperref}
\usepackage{placeins}
\usepackage{graphicx}
\usepackage{subcaption}
\usepackage{physics}




\def\BibTeX{{\rm B\kern-.05em{\sc i\kern-.025em b}\kern-.08em
    T\kern-.1667em\lower.7ex\hbox{E}\kern-.125emX}}


\title{Question 5: Assignment 3: CS 663, Fall 2024}
\author{
\IEEEauthorblockN{
    \begin{tabular}{cccc}
        \begin{minipage}[t]{0.23\textwidth}
            \centering
            Amitesh Shekhar\\
            IIT Bombay\\
            22b0014@iitb.ac.in
        \end{minipage} & 
        \begin{minipage}[t]{0.23\textwidth}
            \centering
            Anupam Rawat\\
            IIT Bombay\\
            22b3982@iitb.ac.in
        \end{minipage} & 
        \begin{minipage}[t]{0.23\textwidth}
            \centering
            Toshan Achintya Golla\\
            IIT Bombay\\
            22b2234@iitb.ac.in
        \end{minipage} \\
        \\ 
    \end{tabular}
}
}

\date{September 24, 2024}


\usepackage{amsmath}
\usepackage{amssymb}
\usepackage{hyperref}
\usepackage{ulem,graphicx}
\usepackage[margin=0.5in]{geometry}

\begin{document}
\maketitle

\\

\begin{enumerate}
\item 
If a function \( f(x, y) \) is real, prove that its Discrete Fourier transform \( F(u, v) \) satisfies \( F^*(u, v) = F(-u, -v) \); if \( f(x, y) \) is real and even, prove that \( F(u, v) \) is also real and even, where \( f(x, y) = f(-x, -y) \). \textsf{[15 points]}
\\
    \makebox[0pt][l]{\hspace{-7pt}\textit{Soln:}} % Aligns "Answer:" to the left
\\
We know that the Discrete Fourier Transform of a discrete function f(x, y) is given by:
\begin{equation}
    F_d(u, v) = \frac{1}{\sqrt{W_1 W_2}} \sum_{x=0}^{W_1-1} \sum_{y=0}^{W_2-1} f(x, y) \exp \left( -j 2 \pi \left( \frac{ux}{W_1} + \frac{vy}{W_2} \right) \right)
\end{equation}
Now, if we were to take conjugate of \( F_d(u, v) \), i.e. \(F_d^*(u, v)\):
\begin{equation}
    F_d^*(u, v) = \frac{1}{\sqrt{W_1 W_2}} \sum_{x=0}^{W_1-1} \sum_{y=0}^{W_2-1} f^*(x, y) \exp \left( j 2 \pi \left( \frac{ux}{W_1} + \frac{vy}{W_2} \right) \right)
\end{equation}

But since f(x, y) is real, as per the question. We can write, f(x, y) = \(f^*(x, y)\) . Modified equation:
\begin{equation}
    F_d^*(u, v) = \frac{1}{\sqrt{W_1 W_2}} \sum_{x=0}^{W_1-1} \sum_{y=0}^{W_2-1} f(x, y) \exp \left( j 2 \pi \left( \frac{ux}{W_1} + \frac{vy}{W_2} \right) \right)
\end{equation}
\begin{equation}
    F_d^*(u, v) = \frac{1}{\sqrt{W_1 W_2}} \sum_{x=0}^{W_1-1} \sum_{y=0}^{W_2-1} f(x, y) \exp \left(- j 2 \pi \left( \frac{- ux}{W_1} + \frac{- vy}{W_2} \right) \right)
\end{equation}
R.H.S of the Equation (4) looks like, Equation with the signs of u and v reversed. Thus, we can say:
\begin{equation}
    F_d^*(u, v) = F_d(-u, -v)
\end{equation}


Next, we want to prove that \(F_d(u, v)\) is real and even given that \(f(x, y)\) is also real and even. \\
Firstly lets try to prove that \(F_d(u, v)\) is real given that \(f(x, y)\) is real and even. We know that the conjugate of \(F_d(u, v)\) i.e. \(F_d^*(u, v)\) is given by: \\
\begin{equation}
    \mathcal{F}_d^*(u, v) = \frac{1}{\sqrt{W_1 W_2}} \sum_{x=0}^{W_1-1} \sum_{y=0}^{W_2-1} f^*(x, y) \exp \left(j 2 \pi \left( \frac{ux}{W_1} + \frac{vy}{W_2} \right) \right)
\end{equation}
But since f(x, y) is real valued and even, we can replace f$^*$(x, y) by f(x, y):
\begin{equation}
    \mathcal{F}_d^*(u, v) = \frac{1}{\sqrt{W_1 W_2}} \sum_{x=0}^{W_1-1} \sum_{y=0}^{W_2-1} f(x, y) \exp \left(j 2 \pi \left( \frac{ux}{W_1} + \frac{vy}{W_2} \right) \right)
\end{equation}
\begin{equation}
    \mathcal{F}_d^*(u, v) = \frac{1}{\sqrt{W_1 W_2}} \sum_{x=0}^{W_1-1} \sum_{y=0}^{W_2-1} f(x, y) \exp \left(-j 2 \pi \left( \frac{-ux}{W_1} + \frac{-vy}{W_2} \right) \right)
\end{equation}
But since, f(x, y) is a even function, f(x, y) = f(-x, -y). Substituting x by $\Tilde{x}$ and y by $\Tilde{y}$, where $\Tilde{x}$ = (-x) and $\Tilde{y}$ = (-y):
\begin{equation}
    \mathcal{F}_d^*(u, v) = \frac{1}{\sqrt{W_1 W_2}} \sum_{\Tilde{x}=0}^{-(W_1-1)} \sum_{\Tilde{y}=0}^{-(W_2-1)} f(\Tilde{x}, \Tilde{y}) \exp \left(-j 2 \pi \left( \frac{u\Tilde{x}}{W_1} + \frac{v\Tilde{y}}{W_2} \right) \right)
\end{equation}
\begin{equation}
    \mathcal{F}_d^*(u, v) = \frac{1}{\sqrt{W_1 W_2}} \sum_{\Tilde{x}=0}^{W_1-1} \sum_{\Tilde{y}=0}^{W_2-1} f(\Tilde{x}, \Tilde{y}) \exp \left(-j 2 \pi \left( \frac{u\Tilde{x}}{W_1} + \frac{v\Tilde{y}}{W_2} \right) \right) = \mathcal{F}_d(u, v)
\end{equation}
Hence $\mathcal{F}_d(u, v)$ is real given f(x, y) is real and even.
\newpage
Now, lets prove that $F_d(u, v)$ is even given that f(x, y) is real and even.
\begin{equation}
    \mathcal{F}_d(-u, -v) = \frac{1}{\sqrt{W_1 W_2}} \sum_{x=0}^{W_1-1} \sum_{y=0}^{W_2-1} f(x, y) \exp \left(-j 2 \pi \left( \frac{-ux}{W_1} + \frac{-vy}{W_2} \right) \right)
\end{equation}
Substituting x by $\Tilde{x}$ and y by $\Tilde{y}$, where $\Tilde{x}$ = (-x) and $\Tilde{y}$ = (-y):
\begin{equation}
    \mathcal{F}_d(-u, -v) = \frac{1}{\sqrt{W_1 W_2}} \sum_{\Tilde{x}=0}^{-(W_1-1)} \sum_{\Tilde{y}=0}^{-(W_2-1)} f(-\Tilde{x}, -\Tilde{y}) \exp \left(-j 2 \pi \left( \frac{u\Tilde{x}}{W_1} + \frac{v\Tilde{y}}{W_2} \right) \right)
\end{equation}
But since f(x, y) is real:
\begin{equation}
    \mathcal{F}_d(-u, -v) = \frac{1}{\sqrt{W_1 W_2}} \sum_{\Tilde{x}=0}^{-(W_1-1)} \sum_{\Tilde{y}=0}^{-(W_2-1)} f(\Tilde{x}, \Tilde{y}) \exp \left(-j 2 \pi \left( \frac{u\Tilde{x}}{W_1} + \frac{v\Tilde{y}}{W_2} \right) \right)
\end{equation}
\begin{equation}
    \mathcal{F}_d(-u, -v) = \frac{1}{\sqrt{W_1 W_2}} \sum_{\Tilde{x}=0}^{W_1-1} \sum_{\Tilde{y}=0}^{W_2-1} f(\Tilde{x}, \Tilde{y}) \exp \left(-j 2 \pi \left( \frac{u\Tilde{x}}{W_1} + \frac{v\Tilde{y}}{W_2} \right) \right) = \mathcal{F}_d(u, v)
\end{equation}
Hence proved that $\mathcal{F}_d(u, v)$ is even, given that f(x, y) is real and even. We can thereby conclude that \(F_d(u, v)\) is real as well as even.\\
\noindent A simpler way to prove that \(F_d(u, v)\) is real and even is using the fact that:
\begin{equation}
    F_d^*(u, v) = F_d(-u, -v)
\end{equation}
Also, we proved that \(F_d(u, v)\) is real above, which means that \(F_d(u, v)\) = \(F_d^*(u, v)\) = \(F_d(-u, -v)\). Clearly, \(F_d(u, v)\) is real and even.
\end{enumerate}

\end{document}
