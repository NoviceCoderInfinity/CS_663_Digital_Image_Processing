\documentclass{article}
\usepackage{helvet}


\usepackage{cite}
\usepackage{amsmath,amssymb,amsfonts}
\usepackage{algorithmic}
\usepackage{graphicx}
\usepackage{textcomp}
\usepackage{xcolor}
\usepackage{hyperref}
\usepackage{placeins}
\usepackage{graphicx}
\usepackage{subcaption}
\usepackage{physics}




\def\BibTeX{{\rm B\kern-.05em{\sc i\kern-.025em b}\kern-.08em
    T\kern-.1667em\lower.7ex\hbox{E}\kern-.125emX}}


\title{Question 7: Assignment 3: CS 663, Fall 2024}
\author{
\IEEEauthorblockN{
    \begin{tabular}{cccc}
        \begin{minipage}[t]{0.23\textwidth}
            \centering
            Amitesh Shekhar\\
            IIT Bombay\\
            22b0014@iitb.ac.in
        \end{minipage} & 
        \begin{minipage}[t]{0.23\textwidth}
            \centering
            Anupam Rawat\\
            IIT Bombay\\
            22b3982@iitb.ac.in
        \end{minipage} & 
        \begin{minipage}[t]{0.23\textwidth}
            \centering
            Toshan Achintya Golla\\
            IIT Bombay\\
            22b2234@iitb.ac.in
        \end{minipage} \\
        \\ 
    \end{tabular}
}
}

\date{September 24, 2024}


\usepackage{amsmath}
\usepackage{amssymb}
\usepackage{hyperref}
\usepackage{ulem,graphicx}
\usepackage[margin=0.5in]{geometry}

\begin{document}
\maketitle

\\

\begin{enumerate}
\item 
Consider the partial differential equation \( \frac{\partial I}{\partial t} = c \left( \frac{\partial^2 I}{\partial x^2} + \frac{\partial^2 I}{\partial y^2} \right) \)
where c is some non-negative constant. This is the isotropic heat equation. Using the differentiation theorem in Fourier transforms, prove that running this PDE on an image I is equivalent to convolving it with a Gaussian of zero mean and appropriate standard deviation. What is the value of the standard deviation? You will also need to use the result that the Fourier transform of a Gaussian is also a Gaussian. [15 points] \textsf{[12+3=15 points]}
\\
    \makebox[0pt][l]{\hspace{-7pt}\textit{Soln:}} % Aligns "Answer:" to the left
\\The heat equation is given as: 
\[
    \frac{\partial I}{\partial t} = c \left( \frac{\partial^2 I}{\partial x^2} + \frac{\partial^2 I}{\partial y^2} \right)
\]
We can consider the image to be a function of x, y and t i.e., I(x, y, t), something like a video clip. Let's assume the 2D Fourier Transform of the image to be $I_F$(u, v, t). We can ignore t, because it doesn't come in the equation of the Fourier Transform. The original equation therefore becomes:
\[
    \frac{\partial I_F(u, v, t)}{\partial t} = c \left( \mathcal{F} \left( \frac{\partial^2 I(x, y, t)}{\partial x^2}\right)+ \mathcal{F} \left( \frac{\partial^2 I(x, y, t)}{\partial y^2} \right) \right)
\]
Simplifying the Right Hand Side of the Equation by using Differentiation Theorem, we get:
\[
    \mathcal{F}\left(\frac{\partial^2 I(x, y, t)}{\partial x^2}\right) = -(2 \pi u)^2 I_F(u, v, t) = - 4\pi^2u^2 I_F(u, v, t)
\]
\[
    \mathcal{F}\left(\frac{\partial^2 I(x, y, t)}{\partial y^2}\right) = -(2 \pi v)^2 I_F(u, v, t) = - 4\pi^2v^2 I_F(u, v, t)
\]

Thus the modified heat equation in context of image becomes:
\[
    \frac{\partial I_F(u, v, t)}{\partial t} = -4\pi^2c(u^2 + v^2)\cdot I_F(u, v, t)
\]
This PDE can be expressed in the solution form as below:
\[
    I_F(u, v, t) = I_F(u, v, 0)\cdot\exp{-4\pi^2ct(u^2+v^2)}
\]
We can use another variable H(u, v, t) in place of the exponential term.
\[
    I_F(u, v, t) = I_F(u, v, 0)\cdot H(u, v, t)
\]
This can be expressed in the time domain as:
\[
    f(x, y, t) = f(x, y, 0) \star h(x, y, t)
\]
That is the solution in domain at time t is the convolution of solution at time (t = 0) and h(x, y, t), which is the inverse Fourier transform of H(u, v, t). Therefore, h(x, y, t) can be computed as:
\[
    \mathcal{F}^{-1}(H(u, v, t)) = \int_{-\infty}^{\infty} \int_{-\infty}^{\infty} H(u, v, t) \exp{2\pi j (ux + vy)} \, du \, dv
\]
\[
    \mathcal{F}^{-1}(H(u, v, t)) = \int_{-\infty}^{\infty} \int_{-\infty}^{\infty} \exp{-4\pi^2ct(u^2+v^2)} \exp{2\pi j (ux + vy)} \, du \, dv
\]
Separating the variables. \\
(From here on, we'll be taking inspiration from the proof of "Fourier of a Gaussian yields a Gaussian")
\[
    \mathcal{F}^{-1}(H(u, v, t)) = \left(\int_{-\infty}^{\infty} \exp{-4\pi^2ctu^2}\, \exp{2\pi jux}\,  du \right) \left(\int_{-\infty}^{\infty} \exp{-4\pi^2ctv^2}\, \exp{2\pi jvy}\,  dv \right)
\]
\[
    \mathcal{F}^{-1}(H(u, v, t)) = \left(\int_{-\infty}^{\infty} \exp{-4\pi^2ct \left(u^2 - \frac{jux}{2\pi c t}\right)}\,  du \right) \left(\int_{-\infty}^{\infty} \exp{-4\pi^2ct \left(v^2 - \frac{jvy}{2\pi c t}\right)}\,  dv \right)
\]
Completing the squares, we get:
\begin{multline*}
    \mathcal{F}^{-1}(H(u, v, t)) = \left( \int_{-\infty}^{\infty} 
    \exp{ -4\pi^2ct \left( \left(u - \frac{jx}{4\pi c t}\right)^2 
    - \left( \frac{jx}{4\pi ct} \right)^2 \right) } du \right) \\
    \times \left( \int_{-\infty}^{\infty} 
    \exp { -4\pi^2ct \left( \left(v - \frac{jy}{4\pi c t}\right)^2 
    - \left( \frac{jy}{4\pi ct} \right)^2 \right) } dv \right)
\end{multline*}
\begin{multline*}
    \mathcal{F}^{-1}(H(u, v, t)) = \left( \int_{-\infty}^{\infty} 
    \exp { -4\pi^2ct \cdot \left(u - \frac{jx}{4\pi c t}\right)^2 
    + 4\pi^2ct \cdot \left( \frac{jx}{4\pi ct} \right)^2 } du \right) \\
    \times \left( \int_{-\infty}^{\infty} 
    \exp { -4\pi^2ct \cdot \left(v - \frac{jy}{4\pi c t}\right)^2 
    + 4\pi^2ct \cdot \left( \frac{jy}{4\pi ct} \right)^2 } dv \right)
\end{multline*}
\begin{multline*}
    \mathcal{F}^{-1}(H(u, v, t)) = \left( \int_{-\infty}^{\infty} 
    \exp { -\left(2\pi\sqrt{ct} \cdot u - 2\pi\sqrt{ct} \cdot \frac{jx}{4\pi c t}\right)^2 
    + \cdot \left( 2\pi\sqrt{ct} \cdot \frac{jx}{4\pi ct} \right)^2 } du \right) \\
    \times \left( \int_{-\infty}^{\infty} 
    \exp { -\left(2\pi\sqrt{ct}v - 2\pi\sqrt{ct}\cdot\frac{jy}{4\pi c t}\right)^2 
    + \left( 2\pi\sqrt{ct}\cdot\frac{jy}{4\pi ct} \right)^2 } dv \right)
\end{multline*}
\begin{multline*}
    \mathcal{F}^{-1}(H(u, v, t)) = \left( \int_{-\infty}^{\infty} 
    \exp { -\left(2\pi\sqrt{ct} \cdot u - \frac{jx}{2\sqrt{ct}}\right)^2 
    + \left( \frac{jx}{2\sqrt{ct}} \right)^2 } du \right) \\
    \times \left( \int_{-\infty}^{\infty} 
    \exp { -\left(2\pi\sqrt{ct}\cdot v - \frac{jy}{2\sqrt{ct}}\right)^2 
    + \left( \frac{jy}{2\sqrt{ct}} \right)^2 } dv \right)
\end{multline*}
\begin{multline*}
    \mathcal{F}^{-1}(H(u, v, t)) = \exp{\left( \frac{jx}{2\sqrt{ct}} \right)^2} \times \left( \int_{-\infty}^{\infty} 
    \exp { -\left(2\pi\sqrt{ct} \cdot u - \frac{jx}{2\sqrt{ct}}\right)^2} du \right) \\
    \times \exp {\left( \frac{jy}{2\sqrt{ct}} \right)^2} \times \left( \int_{-\infty}^{\infty} 
    \exp { -\left(2\sqrt{ct} \cdot v - \frac{jy}{2\sqrt{ct}}\right)^2} dv \right)
\end{multline*}
\begin{multline*}
    \mathcal{F}^{-1}(H(u, v, t)) = \exp{-\frac{x^2 + y^2}{4ct}} \times \left( \int_{-\infty}^{\infty} 
    \exp { -\left(2\pi\sqrt{ct} \cdot u - \frac{jx}{2\sqrt{ct}}\right)^2} du \right) \\
    \times \left( \int_{-\infty}^{\infty} 
    \exp { -\left(2\pi\sqrt{ct} \cdot v - \frac{jy}{2\sqrt{ct}}\right)^2} dv \right)
\end{multline*}
Substituting m = $2\pi u\sqrt{ct} - \frac{jx}{2\sqrt{ct}}$ and n = $2\pi v\sqrt{ct} - \frac{jy}{2\sqrt{ct}}$. We therefore have,
\[
    dm = 2 \pi \sqrt{ct} \, du
\]
\[
    dn = 2 \pi \sqrt{ct} \, dv
\]

Thus the above equation for computing h(x, y, t) becomes
\[
    \mathcal{F}^{-1}(H(u, v, t)) = \exp{-\frac{x^2 + y^2}{4ct}} \times \left( \int_{-\infty}^{\infty} 
    \frac{\exp(-m^2)}{2 \pi \sqrt{ct}} dm \right) \times \left( \int_{-\infty}^{\infty} 
    \frac{\exp(-n^2)}{2 \pi \sqrt{ct}} dn \right)
\]
Since, we know that $\int_{-\infty}^{\infty}exp(-x^2) dx = \sqrt{\pi}$, we can resolve the above equation into:
\[
    \mathcal{F}^{-1}(H(u, v, t)) = \exp{-\frac{x^2 + y^2}{4ct}} \times \frac{\sqrt{\pi}}{2 \pi \sqrt{ct}} \times \frac{\sqrt{\pi}}{2 \pi \sqrt{ct}} = \frac{1}{4 \pi ct}\cdot\exp{-\frac{x^2 + y^2}{4ct}} = h(x, y, t)
\]
We know that equation of a standard gaussian in 2D is given as,
\[
    g(x, y) = \frac{1}{2\pi\sigma^2} \cdot \exp{-\frac{(x - \mu_x)^2 + (y - \mu_y)^2}{2\sigma^2}}
\]
Taking mean as zero, it reduces to,
\[
    g(x, y) = \frac{1}{2\pi\sigma^2} \cdot \exp{-\frac{x^2 + y^2}{2\sigma^2}}
\]
Hence proved that running this PDE on an image I is equivalent to convolving the value of I at time t = 0 with a Gaussian of zero mean and some standard deviation. \\ \\ 

Comparing the above equation of gaussian in 2D with the obtained form for h(x, y, t), we can say that standard deviation, $\sigma = \sqrt{2ct}$ increases over time. Eventually, as $t \to \infty$, the image becomes a constant value, similar to the way heat distribution reaches a constant value over time.

\end{enumerate}

\end{document}
