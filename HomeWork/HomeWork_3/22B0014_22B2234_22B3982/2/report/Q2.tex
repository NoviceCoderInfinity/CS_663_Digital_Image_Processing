\documentclass{article}
\usepackage{helvet}


\usepackage{cite}
\usepackage{amsmath,amssymb,amsfonts}
\usepackage{algorithmic}
\usepackage{graphicx}
\usepackage{textcomp}
\usepackage{xcolor}
\usepackage{hyperref}
\usepackage{placeins}
\usepackage{graphicx}
\usepackage{subcaption}
\usepackage{physics}




\def\BibTeX{{\rm B\kern-.05em{\sc i\kern-.025em b}\kern-.08em
    T\kern-.1667em\lower.7ex\hbox{E}\kern-.125emX}}


\title{Question 2: Assignment 3: CS 663, Fall 2024}
\author{
\IEEEauthorblockN{
    \begin{tabular}{cccc}
        \begin{minipage}[t]{0.23\textwidth}
            \centering
            Amitesh Shekhar\\
            IIT Bombay\\
            22b0014@iitb.ac.in
        \end{minipage} & 
        \begin{minipage}[t]{0.23\textwidth}
            \centering
            Anupam Rawat\\
            IIT Bombay\\
            22b3982@iitb.ac.in
        \end{minipage} & 
        \begin{minipage}[t]{0.23\textwidth}
            \centering
            Toshan Achintya Golla\\
            IIT Bombay\\
            22b2234@iitb.ac.in
        \end{minipage} \\
        \\ 
    \end{tabular}
}
}

\date{September 24, 2024}


\usepackage{amsmath}
\usepackage{amssymb}
\usepackage{fdsymbol}
\usepackage{bbding}
\usepackage{fontawesome}
\usepackage{pifont}
\usepackage{hyperref}
\usepackage{ulem,graphicx}
\usepackage[margin=0.5in]{geometry}

\begin{document}
\maketitle

\\

\begin{enumerate}
\item 
Derive the 2D Fourier transform of the correlation of two continuous 2D signals in the continuous domain.
Repeat the same for the 2D DFT of two 2D discrete signals. \textsf{[10 points]}
\\
    \makebox[0pt][l]{\hspace{-7pt}\textit{Soln:}} % Aligns "Answer:" to the left
\\
The Correlation Theorem states that:

\[
f(x, y) \text{\ding{73}} h(x, y) \Longleftrightarrow F^*(u, v) H(u, v)
\]
\[
f^*(x, y) h(x, y) \Longleftrightarrow F(u, v) \text{\ding{73}} H(u, v)
\]
whereas, the \textbf{correlation operation for continous domain} is given as:
\[
f(x, y) \text{\ding{73}} h(x, y) = \int_{-\infty}^{\infty} \int_{-\infty}^{\infty} f^*(m, n) \cdot h(x + m, y + n) \cdot dm \cdot dn
\]
The \textit{Fourier Transform of the convolution of two 2D functions f(x, y) and h(x, y) is given as}:
\[
   F(f(x, y) \text{\ding{73}} h(x, y)) = \int_{-\infty}^{\infty} \int_{-\infty}^{\infty} \left( \int_{-\infty}^{\infty} \int_{-\infty}^{\infty} f^*(m, m) h(x + m, y + n) \, dm \, dn \right) \exp(-2 \pi j (ux + vy)) \, dx \, dy
\]
\[
   F(f(x, y) \text{\ding{73}} h(x, y)) = \int_{-\infty}^{\infty} \int_{-\infty}^{\infty} f^*(m, m) \left( \int_{-\infty}^{\infty} \int_{-\infty}^{\infty} h(x + m, y + n) \, \exp(-2 \pi j (ux + vy)) \, dx \, dy \right) \, dm \, dn
\]
Let $\tilde{x}$ = x + m, $\tilde{y}$ = y + n. Substituting them accordingly in the inner integral, we get:
\[
   F(f(x, y) \text{\ding{73}} h(x, y)) = \int_{-\infty}^{\infty} \int_{-\infty}^{\infty} f^*(m, m) \left( \int_{-\infty}^{\infty} \int_{-\infty}^{\infty} h(\tilde{x}, \tilde{y}) \, \exp(-2 \pi j (u(\tilde{x} - m) + v(\tilde{y} - n))) \, d\tilde{x} \, d\tilde{y} \right) \, dm \, dn
\]
\[
   F(f(x, y) \text{\ding{73}} h(x, y)) = \int_{-\infty}^{\infty} \int_{-\infty}^{\infty} f^*(m, m) \, \exp(2 \pi j (um + vn)) \, \left( \int_{-\infty}^{\infty} \int_{-\infty}^{\infty} h(\tilde{x}, \tilde{y}) \, \exp(-2 \pi j (u\tilde{x} + v\tilde{y})) \, d\tilde{x} \, d\tilde{y} \right) \, dm \, dn
\]
The inner integral of h($\tilde{x}$, $\tilde{y}$) evaluates to its Fourier Transform H(u, v):
\[
   F(f(x, y) \text{\ding{73}} h(x, y)) = \int_{-\infty}^{\infty} \int_{-\infty}^{\infty} f^*(m, m) \, \exp(2 \pi j (um + vn)) \, H(u, v) \, dm \, dn
\]
\[
   F(f(x, y) \text{\ding{73}} h(x, y)) = H(u, v) \int_{-\infty}^{\infty} \int_{-\infty}^{\infty} f^*(m, m) \, \exp(2 \pi j (um + vn)) \, dm \, dn
\]
Since, we know that Fourier Transform a 2D function g(x, y) is given by:
\[
   F(g(x, y))(u, v) = \int_{-\infty}^{\infty} \int_{-\infty}^{\infty} g(x, y) \, \exp(-2 \pi j (ux + vy)) \, dx \, dy
\]
\[
   F^*(g(x, y))(u, v) = \int_{-\infty}^{\infty} \int_{-\infty}^{\infty} g^*(x, y) \, \exp(+2 \pi j (ux + vy)) \, dx \, dy
\]
Thus, Fourier Transform of the correlation of f(x, y) and h(x, y) becomes: 
\[
   F(f(x, y) \text{\ding{73}} h(x, y)) = F^*(u, v) H(u, v)
\]
This proves the first part of the correlation theorem for continuous domain.

\newpage

\textit{The inverse Fourier Transform of correlation of F(u, v) and H(u, v) is given as}:
\[
    F^{-1}(F(u, v) \text{\ding{73}} H(u, v)) = \int_{-\infty}^{\infty} \int_{-\infty}^{\infty} \left( \int_{-\infty}^{\infty} \int_{-\infty}^{\infty} F^*(m, n) \cdot H(u + m, v + n) \cdot dm \cdot dn \right) \exp(2 \pi j (ux + vy)) \, du \, dv
\]
\[
    F^{-1}(F(u, v) \text{\ding{73}} H(u, v)) = \int_{-\infty}^{\infty} \int_{-\infty}^{\infty} F^*(m, n) \left( \int_{-\infty}^{\infty} \int_{-\infty}^{\infty} H(u + m, v + n) \exp(2 \pi j (ux + vy)) \cdot du \cdot dv \right) \, dm \, dn
\]
Let $\tilde{u}$ = u + m and $\tilde{v}$ = v + n. Substituting it in the inner integral, we get:
\[
    F^{-1}(F(u, v) \text{\ding{73}} H(u, v)) = \int_{-\infty}^{\infty} \int_{-\infty}^{\infty} F^*(m, n) \left( \int_{-\infty}^{\infty} \int_{-\infty}^{\infty} H(\tilde{u}, \tilde{v}) \exp(2 \pi j ((\tilde{u} - m)x + (\tilde{v} - n)y)) \cdot d\tilde{u} \cdot d\tilde{v} \right) \, dm \, dn
\]
\[
    F^{-1}(F(u, v) \text{\ding{73}} H(u, v)) = \int_{-\infty}^{\infty} \int_{-\infty}^{\infty} F^*(m, n) \exp(- 2 \pi j (mx + ny)) \left( \int_{-\infty}^{\infty} \int_{-\infty}^{\infty} H(\tilde{u}, \tilde{v}) \exp(2 \pi j (\tilde{u}x + \tilde{v}y)) \cdot d\tilde{u} \cdot d\tilde{v} \right) \, dm \, dn
\]
The inner integral evaluates to the inverse Fourier transform of H($\tilde{u}$, $\tilde{v}$), i.e. h(x, y):
\[
    F^{-1}(F(u, v) \text{\ding{73}} H(u, v)) = \int_{-\infty}^{\infty} \int_{-\infty}^{\infty} F^*(m, n) \exp(- 2 \pi j (mx + ny)) h(x, y) \, dm \, dn
\]
\[
    F^{-1}(F(u, v) \text{\ding{73}} H(u, v)) = h(x, y) \int_{-\infty}^{\infty} \int_{-\infty}^{\infty} F^*(m, n) \exp(- 2 \pi j (mx + ny)) \, dm \, dn
\]
The inverse Fourier Transform g(x, y) of the function G(u, v) is given by:
\[
    g(x, y) = \int_{-\infty}^{\infty} \int_{-\infty}^{\infty} G(m, n) \exp(2 \pi j (mx + ny)) \, dm \, dn
\]
\[
    g^*(x, y) = \int_{-\infty}^{\infty} \int_{-\infty}^{\infty} G^*(m, n) \exp(-2 \pi j (mx + ny)) \, dm \, dn
\]

The inverse Fourier Transform of correlation of F(u, v) and H(u, v) is given as:
\[
    F^{-1}(F(u, v) \text{\ding{73}} H(u, v)) = f^*(x, y) h(x, y)
\]


Next, we compute the same result for Discrete signals domain:
The Correlation Theorem states that:

\[
f(x, y) \text{\ding{73}} h(x, y) \Longleftrightarrow F^*(u, v) H(u, v)
\]
\[
f^*(x, y) h(x, y) \Longleftrightarrow F(u, v) \text{\ding{73}} H(u, v)
\]
whereas, the \textbf{correlation operation for discrete domain} is given as:
\[
f(x, y) \text{\ding{73}} h(x, y) = \sum_{m=0}^{M-1} \sum_{n=0}^{N-1} f^*(m, n) \cdot h(x + m, y + n)
\]
The \textit{Fourier Transform of the correlation of two signals f(x, y) and h(x, y)} is:
\[
\mathcal{F}(f(x, y) \text{\ding{73}} h(x, y))(u, v) = \sum_{x=0}^{M-1} \sum_{y=0}^{N-1} \left( \sum_{m=0}^{M-1} \sum_{n=0}^{N-1} f^*(m, n) \cdot h(x + m, y + n) \right) \exp(-2 \pi j (ux/N + vy/M)) 
\]
\[
\mathcal{F}(f(x, y) \text{\ding{73}} h(x, y))(u, v) = \sum_{x=0}^{M-1} \sum_{y=0}^{N-1} f^*(m, n) \left( \sum_{m=0}^{M-1} \sum_{n=0}^{N-1} h(x + m, y + n) \exp(-2 \pi j (ux/N + vy/M)) \right)
\]
Let $\tilde{x}$ = x + m and $\tilde{y}$ = y + n. Substituting this in the above equation, we get:
\[
\mathcal{F}(f(x, y) \text{\ding{73}} h(x, y))(u, v) = \sum_{x=0}^{M-1} \sum_{y=0}^{N-1} f^*(m, n) \left( \sum_{m=0}^{M-1} \sum_{n=0}^{N-1} h(\tilde{x}, \tilde{y}) \exp(-2 \pi j (u(\tilde{x} - m)/N + v(\tilde{y} - n)/M)) \right)
\]
\[
\mathcal{F}(f(x, y) \text{\ding{73}} h(x, y))(u, v) = \sum_{x=0}^{M-1} \sum_{y=0}^{N-1} f^*(m, n) \exp(2 \pi j (um/N + vn/M)) \left( \sum_{m=0}^{M-1} \sum_{n=0}^{N-1} h(\tilde{x}, \tilde{y}) \exp(-2 \pi j (u\tilde{x}/N + v\tilde{y}/M)) \right)
\]
\newpage
The inner summation over h($\tilde{x}$, $\tilde{y}$) evaluates to the Fourier Transform of h(x, y) which is H(u, v):
\[
\mathcal{F}(f(x, y) \text{\ding{73}} h(x, y))(u, v) = \sum_{x=0}^{M-1} \sum_{y=0}^{N-1} f^*(m, n) \exp(2 \pi j (um/N + vn/M)) H(u, v)
\]
\[
\mathcal{F}(f(x, y) \text{\ding{73}} h(x, y))(u, v) = H(u, v) \sum_{x=0}^{M-1} \sum_{y=0}^{N-1} f^*(m, n) \exp(2 \pi j (um/N + vn/M))
\]
Now the summation evaluates to the Fourier Transform of f(m, n) which evaluates to $F^*(u, v)$:
\[
\mathcal{F}(f(x, y) \text{\ding{73}} h(x, y))(u, v) = F^*(u, v) H(u, v)
\]

\textit{The inverse Fourier Transform of correlation of F(u, v) and H(u, v) is given as}:
\[
    \mathcal{F}^{-1}(F(u, v) \text{\ding{73}} H(u, v)) = \sum_{u=0}^{M-1} \sum_{v=0}^{N-1} \left( \sum_{u=0}^{M-1} \sum_{v=0}^{N-1} F^*(m, n) \cdot H(u + m, v + n) \right) \exp(2 \pi j (ux/M + vy/N))
\]
\[
    \mathcal{F}^{-1}(F(u, v) \text{\ding{73}} H(u, v)) = \sum_{u=0}^{M-1} \sum_{v=0}^{N-1} F^*(m, n) \left( \sum_{u=0}^{M-1} \sum_{v=0}^{N-1} H(u + m, v + n) \exp(2 \pi j (ux / M+ vy/N)) \right)
\]

Let $\tilde{u}$ = u + m and $\tilde{v}$ = v + n. Substituting it in the inner summation, we get:
\[
    \mathcal{F}^{-1}(F(u, v) \text{\ding{73}} H(u, v)) = \sum_{u=0}^{M-1} \sum_{v=0}^{N-1} F^*(m, n) \left( \sum_{\tilde{u}=0}^{M-1} \sum_{\tilde{v}=0}^{N-1} H(\tilde{u}, \tilde{v}) \exp(2 \pi j ((\tilde{u} - m)x/M + (\tilde{v} - n)y/N)) \right)
\]
\[
    \mathcal{F}^{-1}(F(u, v) \text{\ding{73}} H(u, v)) = \sum_{u=0}^{M-1} \sum_{v=0}^{N-1} F^*(m, n) \exp(-2 \pi j (mx/M + ny/N)) \left( \sum_{\tilde{u}=0}^{M-1} \sum_{\tilde{v}=0}^{N-1} H(\tilde{u}, \tilde{v}) \exp(2 \pi j (\tilde{u}x/M + \tilde{v}y/N)) \right)
\]
The inner summation evaluates to the inverse fourier transform of H($\tilde{u}$, $\tilde{v}$), i.e., h(x, y)
\[
    \mathcal{F}^{-1}(F(u, v) \text{\ding{73}} H(u, v)) = \sum_{u=0}^{M-1} \sum_{v=0}^{N-1} F^*(m, n) \exp(-2 \pi j (mx/M + ny/N)) \, h(x, y)
\]
\[
    \mathcal{F}^{-1}(F(u, v) \text{\ding{73}} H(u, v)) = h(x, y) \sum_{u=0}^{M-1} \sum_{v=0}^{N-1} F^*(m, n) \exp(-2 \pi j (mx/M + ny/N))
\]
The sumamtion evaluates to the conjugate of f(x, y):
\[
    \mathcal{F}^{-1}(F(u, v) \text{\ding{73}} H(u, v)) = f^*(x, y) \, h(x, y)
\]

Hence the correlation theorem is proved for both continous and discrete domains in 2D.
\end{enumerate}
\end{document}
